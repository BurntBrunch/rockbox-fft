% $Id$ %
\chapter{The Main Menu}
\section{\label{ref:main_menu}Introducing the Main Menu}
\screenshot{main_menu/images/ss-main-menu}{The main menu}{}
The \setting{Main Menu} is the screen from which all of the Rockbox functions
can be accessed. This is the first screen you will see when starting Rockbox.
To return to the \setting{Main Menu}, 
  \nopt{ONDIO_PAD}{press the \ActionStdMenu{} button.}%
  \opt{ONDIO_PAD}{hold the \ButtonMenu{} button.}%

All settings are stored on the unit. However, Rockbox does not spin up 
the disk solely for the purpose of saving settings. Instead, Rockbox will
save settings when it spins up the disk the next time, for example when 
refilling the MP3 buffer or navigating through the \setting{File Browser}.
Changes to settings may therefore not be saved unless the \dap{} is shut down
safely (see \reference{ref:Safeshutdown}).

\section{Navigating the Main Menu}
\begin{table}
  \begin{btnmap}{}{}
    \ActionStdNext
    \opt{HAVEREMOTEKEYMAP}{& \ActionRCStdNext}    
    & Select the next option in the menu.\\
    
    & \opt{HAVEREMOTEKEYMAP}{&} Inside a setting,
      increase the value or choose next option. \\
    %
    \ActionStdPrev
    \opt{HAVEREMOTEKEYMAP}{& \ActionRCStdPrev}
    & Select the previous option in the menu.\\
    & \opt{HAVEREMOTEKEYMAP}{&} Inside a setting,
      decrease the value or choose previous option. \\
    %
    \ActionStdOk
    \opt{HAVEREMOTEKEYMAP}{& \ActionRCStdOk}
    & Select option. \\
    %
    \ActionStdCancel
    \opt{HAVEREMOTEKEYMAP}{& \ActionRCStdCancel}
    & Exit menu or setting, or move to parent menu.\\
  \end{btnmap}
\end{table}

\section {Recent Bookmarks}
\screenshot{main_menu/images/ss-list-bookmarks}%
{The list bookmarks screen}{}
If the \setting{Save a list of recently created bookmarks} option is enabled 
then you can view a list of several recent bookmarks here and select one to 
jump straight to that track.\\*

 \note{Bookmarking only works when tracks are launched from the file browser,
        and does not currently work for tracks launched via the
        database. In addition, they do not currently work with dynamic
        playlists.\\*} 

\begin{table}
  \begin{btnmap}{}{}
    \ActionStdNext
    \opt{HAVEREMOTEKEYMAP}{& \ActionRCStdNext}
    & Select the next bookmark.\\
    %
    \ActionStdPrev
    \opt{HAVEREMOTEKEYMAP}{& \ActionRCStdPrev}
    & Select the previous bookmark.\\
    %
    \ActionStdOk
    \opt{HAVEREMOTEKEYMAP}{& \ActionRCStdOk}
    & Resume from the selected bookmark.\\
    %
    \ActionStdCancel
    \opt{HAVEREMOTEKEYMAP}{& \ActionRCStdCancel}
    & Exit Recent Bookmark menu.\\
    %
    \nopt{GIGABEAT_S_PAD}{\ActionBmDelete
    \opt{HAVEREMOTEKEYMAP}{& \ActionRCBmDelete}
    & Delete the currently selected bookmark.\\}
    %
    \ActionStdContext
    \opt{HAVEREMOTEKEYMAP}{& \ActionRCStdContext}  
    & Enter the context menu for the selected bookmark.\\
  \end{btnmap}
\end{table}

There are two options in the context menu:\\*
  
  \setting{Resume} will commence playback of the currently selected bookmark entry.
  
  \setting{Delete} will remove the currently selected bookmark entry from the list.\\*
  
This entry is not shown in the \setting{Main Menu} when the option is off
(the default setting).  See \reference{ref:Bookmarkconfigactual} 
for more details on configuring bookmarking in Rockbox.

\section{Files}
Browse the files on your \dap{} (see \reference{ref:file_browser}).

\section{Database}
Browse by the meta-data in your audio files (see \reference{ref:database}).

\section{Now Playing/Resume Playback}
Go to the \setting{While Playing Screen} and resume if music playback is
stopped or paused and there is something to resume (see \reference{ref:WPS}).

\section{Settings}

The \setting{Settings} menu allows you to set or adjust many parameters that
affect the way your \dap{} works. There are many submenus for different
parameter areas. Every time you are setting a value of a parameter, and that
value is selected from a list of some predefined available values, you can press
\ActionStdContext, and the selection cursor will jump to the default value for
the parameter. You can then confirm or cancel the value. This is useful if you
have changed the value of the parameter from the default to some other value and
would like to restore the default value.

\subsection{Sound Settings}
The \setting{Sound Settings} menu offers a selection of sound properties you may 
change to customise your listening experience. The details of this menu are covered
in \reference{ref:configure_rockbox_sound}.

\subsection{Playback Settings}
The \setting{Playback Settings} menu allows you to configure settings related
to audio playback. The details of this menu are covered
in \reference{ref:configure_rockbox_playback}.

\subsection{General Settings}
The \setting{General Settings} menu allows you to customise the way Rockbox looks 
and the way it plays music. The details of this menu are covered in
\reference{ref:configure_rockbox_general}.

\subsection{Theme Settings}
The \setting{Theme Settings} menu contains options that control the visual
apperance of Rockbox. The details of this menu are covered in
\reference{ref:configure_rockbox_themes}.

\opt{recording}{
\subsection{Recording Settings}
The \setting{Recording Settings} menu allows you to configure settings related
to recording. The details of this menu are covered in detail in
\reference{ref:Recordingsettings}.
}

\subsection{Manage Settings}
The \setting{Manage Settings} option allows the saving and re-loading of user 
configuration settings, browsing the hard drive for alternate firmwares, and finally
resetting your \dap{} back to initial configuration.
%
The details of this menu are covered in
\reference{ref:manage_settings}.

\opt{recording}{% $Id$ %
\section{\label{ref:Recording}Recording}
\subsection{\label{ref:while_recording_screen}While Recording Screen}
\screenshot{main_menu/images/ss-while-recording-screen}{The while recording
  screen}{}

Selecting the \setting{Recording} option in the \setting{Main Menu} enters
the \setting{Recording Screen}, whilst pressing \ActionStdContext{} enters the
\setting{Recording Settings} (see \reference{ref:Recordingsettings}). 
The \setting{Recording Screen}
shows the time elapsed and the size of the file being recorded. A peak meter
is present to allow you set gain correctly. There is also a volume setting,
this will only affect the output level of the \dap{} and does \emph{not}
affect the recorded sound. If enabled in the peak meter settings, a counter in
front of the peak meters shows the number of times the clip indicator was
activated during recording. The counter is reset to zero when starting a new
recording.\\*

\opt{recording_swcodec}{
\opt{disk_storage}{
\note{When you start a recording, the hard disk will spin up. This will cause
the peak meters to freeze in the process. This is expected behaviour, and
nothing to worry about. The recording continues during the spin up.\\*}}}

\opt{recording_hwcodec}{The frequency, channels and quality}
\opt{recording_swcodec}{The frequency and channels} settings are shown in the
status bar.\\*

The controls for this screen are:
\begin{table}
  \begin{btnmap}{}{}
    
    \ActionRecPrev{} / \ActionRecNext 
    \opt{HAVEREMOTEKEYMAP}{& \ActionRCRecPrev{} / \ActionRCRecNext}
    & Select setting.\\
    %
    \ActionRecSettingsDec{} / \ActionRecSettingsInc
    \opt{HAVEREMOTEKEYMAP}{& \ActionRCRecSettingsDec{} / \ActionRecSettingsInc}
    & Adjust selected setting.\\
    %
    \ActionRecPause
        &
    \opt{HAVEREMOTEKEYMAP}{
        \ActionRCRecPause
        &}
    Start recording.\newline
    While recording: pause recording (press again to continue).\\
    %
    \ActionRecExit
        &
    \opt{HAVEREMOTEKEYMAP}{
        \ActionRCRecExit
        &}
    Exit \setting{Recording Screen}.\newline
    While recording: Stop recording.
        \\
    %
    \opt{IRIVER_H10_PAD,IRIVER_H100_PAD,IRIVER_H300_PAD,IAUDIO_X5_PAD%
        ,SANSA_E200_PAD,IPOD_4G_PAD,SANSA_C200_PAD}{
        \ActionRecNewfile
            &
        \opt{HAVEREMOTEKEYMAP}{
            \ActionRCRecNewfile
            &}
        Start recording.\newline
        While recording: close the current file and open a new one.
            \\
        }
    %
    \ActionRecMenu
    \opt{HAVEREMOTEKEYMAP}{& \ActionRCRecMenu}
                       & Open \setting{Recording Settings} (see 
                       \reference{ref:Recordingsettings}).\\
    %
    \opt{RECORDER_PAD}{
      \ActionRecFTwo & Quick menu for recording settings. A quick press will
      leave the screen up (press \ActionRecFTwo{} again to exit), while holding
      it will close the screen when you release it.\\
    }
    %
    \opt{RECORDER_PAD}{
      \ActionRecFThree & Quick menu for source setting.\\
      & Quick/hold works as for \ActionRecFTwo.\\
      & While recording: Start a new recording file.\\
    }
  \end{btnmap}
\end{table}
}

\opt{radio}{% $Id$ %
\section{\label{ref:FMradio}FM Radio}  
\opt{RECORDER_PAD}{
  \note{The early V2 models were in fact FM Recorders in disguise,
  so they had the FM radio still mounted. Rockbox enables it if present -
  in case this menu does not show on your unit you can skip this chapter.\\}
}
\opt{sansa}{
  \note{Not all Sansas have a radio receiver. Generally all american models do,
  but european models might not. Rockbox will display the radio menu only if it
  can find a radio receiver in your Sansa.}
}

\screenshot{main_menu/images/ss-fm-radio-screen}{The FM radio screen}{}
  This menu option switches to the radio screen.
  The FM radio has the ability to remember station frequency settings
  (presets). Since stations and their frequencies vary depending on location,
  it is possible to load these settings from a file. Such files should have
  the filename extension \fname{.fmr} and reside in the directory
  \fname{/.rockbox/fmpresets} (note that this directory does not exist after
  the initial Rockbox installation; you should create it manually). To load
  the settings, i.e. a set of FM stations, from a preset file, just ``play''
  it from the file browser. Rockbox will ``remember'' and use it in
  \setting{PRESET} mode until another file has been selected. Some preset
  files are available here: \wikilink{FmPresets}
  
  \opt{recording}{
      \opt{swcodec}{
         It is also possible to record the FM radio while listening.
         To start recording, enter the FM radio settings menu with
         \ActionFMMenu{} and then select \setting{Recording}.
         At this point, you will be switched to the \setting{Recording Screen}.
         Further information on \setting{Recording} can be found in
         \reference{ref:Recording}.
      }
  }

  \opt{masf}{\note{The radio will shorten battery life, because the
      MAS-chip is set to record mode for instant recordings.}
  }

    \begin{table}
      \begin{btnmap}{}{}
          \ActionFMPrev, \ActionFMNext
          \opt{HAVEREMOTEKEYMAP}{& \ActionRCFMPrev, \ActionRCFMNext}
          & Change frequency in \setting{SCAN} mode or jump to next/previous
          station in \setting{PRESET} mode\\
          %
          Long \ActionFMPrev, \ActionFMNext
          \opt{HAVEREMOTEKEYMAP}{& Long \ActionRCFMPrev, Long \ActionRCFMNext}
          & Seek to next station in \setting{SCAN} mode.\\
          %
          \ActionFMSettingsInc, \ActionFMSettingsDec
          \opt{HAVEREMOTEKEYMAP}{
              &
              \opt{IRIVER_RC_H100_PAD}{\ActionRCFMVolUp, \ActionRCFMVolDown}%
              \nopt{IRIVER_RC_H100_PAD}{\ActionRCFMSettingsInc, \ActionRCFMSettingsDec}%
          }
          & Change volume.\\
          \opt{RECORDER_PAD}{
            \ButtonPlay
            & Freezes all screen updates. May enhance radio reception in some
              cases.\\
          }

          %
          \ActionFMExit
          \opt{HAVEREMOTEKEYMAP}{& \ActionRCFMExit}
          & Leave the radio screen with the radio playing.\\
          %
          \ActionFMStop
          \opt{HAVEREMOTEKEYMAP}{& \ActionRCFMStop}
          & Stops the radio and returns to \setting{Main Menu}.\\%
          %
          \nopt{ONDIO_PAD}{%
            \nopt{RECORDER_PAD}{\ActionFMPlay
              \opt{HAVEREMOTEKEYMAP}{& \ActionRCFMPlay}
              & Mutes radio playback.\\}%
            %
            \ActionFMMode
            \opt{HAVEREMOTEKEYMAP}{& \ActionRCFMMode}
            & Switches between \setting{SCAN} and \setting{PRESET} mode.\\
            %
            \ActionFMPreset
            \opt{HAVEREMOTEKEYMAP}{& \ActionRCFMPreset}
            & Opens a list of radio presets. You can view all the presets that 
              you have, and switch to the station.\\
          }%
          %
          \ActionFMMenu
          \opt{HAVEREMOTEKEYMAP}{& \ActionRCFMMenu}
          & Displays the FM radio settings menu.\\
       \end{btnmap}
    \end{table}

  \begin{description}

  \item[Saving a preset:]
    Up to 64 of your favourite stations can be saved as presets.
    \opt{RECORDER_PAD}{Press \ButtonFTwo{} to go to the presets list, press
    \ButtonFOne{} to add a preset.}%
    \nopt{RECORDER_PAD}{%
      \ActionFMMenu{} to go to the menu, then select \setting{Add preset}.%
    }
    Enter the name (maximum number of characters is 32).
    Press \ActionKbdDone{} to save.

  \item[Selecting a preset:]
        \opt{ONDIO_PAD}{\ActionFMMenu{} to open the menu, select
          \setting{Preset}}%
        \nopt{ONDIO_PAD}{\ActionFMPreset} to go to the presets list.
        Use \ActionFMSettingsInc{} and \ActionFMSettingsDec{}
        to move the cursor and then press \ActionStdOk{}
        to select. Use \ActionStdCancel{} to leave the preset list without selecting
        anything.

  \item[Removing a preset:]
        \opt{ONDIO_PAD}{\ActionFMMenu{} to open the menu, select
          \setting{Preset}}%
        \nopt{ONDIO_PAD}{\ActionFMPreset} to go to the presets list.
        Use \ActionFMSettingsInc{} and \ActionFMSettingsDec{}
        to move the cursor and then press \ActionStdContext{}
        on the preset that you wish to remove, then select \setting{Remove Preset}.

      \opt{RECORDER_PAD,ONDIO_PAD}{
          \item[Recording:]
            \opt{RECORDER_PAD}{Press \ButtonFThree}%
            \opt{ONDIO_PAD}{Double press \ButtonMenu}
            to start recording the currently playing station. Press \ButtonOff{} to
              stop recording.%
            \opt{RECORDER_PAD}{ Press \ButtonPlay{} again to seamlessly start recording
              to a new file.}
            The settings for the recording can be changed in the
            \opt{RECORDER_PAD}{\ButtonFOne{} menu}%
            \opt{ONDIO_PAD}{respective menu reached through the FM radio settings menu
              (Long \ButtonMenu)}
            before starting the recording. See \reference{ref:Recordingsettings}
            for details of recording settings.
          }
  \end{description}
  \note{The radio will turn off when starting playback of an audio file.}
}

\section{\label{ref:playlistoptions}Playlist}
  This menu allows you to work with playlists. Playlists can be created in 
  three ways. Playing a file in a directory causes all the files in it
  to be placed in a playlist. Playlists can be created manually by
  either using the  \setting{Context Menu} (see \reference{ref:Contextmenu}) or using
  the \setting{Playlist} menu. Both automatically and manually created
  playlists can be edited using this menu.

\begin{description}
\item[Create Playlist:]
  Rockbox will create a playlist with all tracks in the current directory 
and all sub-directories. The playlist will be created one directory level ``up'' 
from where you currently are.
  
\item[View Current Playlist:]
  Displays the contents of the playlist currently stored in memory.
  
\item[Save Current Playlist:]
  Saves the current dynamic playlist, excluding queued tracks, to the 
specified file. If no path is provided then playlist is saved to the current 
directory.

\item[Playlist Catalog:]
  The \setting{Playlist Catalog} provides a simple interface to maintain
  several playlists (see \reference{ref:working_with_playlists}).
\end{description}

\section{Plugins}
  With this option you can load and run various plugins that have been
written for Rockbox. There are a wide variety of these supplied with
Rockbox, including several games, some impressive demos and a number of
utilities. A detailed description of the different plugins is to be found in 
\reference{ref:plugins}.

\section{\label{ref:Info}System}
\opt{player}{Use the MINUS and PLUS keys to step through several 
pages of information.}

\begin{description}
\opt{rtc}{
  \item[Time and Date:]
    Time related menu options. Pressing \ActionStdContext{} will voice the current time if voice support is enabled
    \begin{description}
      \item [Set Time/Date: ] Set current time and date.
      \item[Sleep Timer:]
        The \setting{Sleep Timer} powers off your \dap{} after playing for a given
        time. It can be set from \setting{Off} to 5 hours in 5 minute steps.
        The \setting{Sleep Timer} is reset on boot.
        \opt{alarm}{Using this option disables the \setting{Wake up alarm}.}
      \item [Time Format: ] Choose 12 or 24 hour clock.
      \opt{alarm}{
        \subsection{Wake-Up Alarm}
        % TODO this isn't quite right for all targets, I think
            This option turns the \dap{} off and then starts it up again at the
            specified time. Use \ActionSettingInc{} and \ActionSettingDec{} to adjust
            the minutes setting, \ActionAlarmHoursDec{} and \ActionAlarmHoursInc{} to
            adjust the hours.
            \ActionAlarmSet{} confirms the alarm and shuts the \dap{} down, and
            \ActionAlarmCancel{} cancels setting an alarm. If the \dap{} is turned on
            again before the alarm occurs, the alarm will be cancelled. Using this
            option disables the \setting{Sleep Timer}.

        \opt{recording,radio}{
            \subsection{Alarm Wake up Screen}
            This option controls what the \dap{} does when it is woken up by the alarm.
        }
      }
    \end{description}
}
\item[Rockbox Info:]
  Displays some basic system information. This is, from top to bottom,
  the amount of memory Rockbox has available for storing music (the buffer).
  The battery status.
\opt{multivolume}{%
  Memory size and amount of free space on the two data volumes, this info is
  given seperately for internal memory (\emph{Int}) and for a plugged in
  memory card
  \opt{ondio}{(\emph{MMC})}
  \opt{sansa,e200v2,fuze}{(\emph{MSD})}.
}%
\nopt{multivolume}{Hard disk size and the amount of free space on the disk.}

\item[Credits:]
  Display the list of contributors.

\nopt{rtc}{
  \item[Sleep Timer:]
    The \setting{Sleep Timer} powers off your \dap{} after playing for a given
    time. It can be set from \setting{Off} to 5 hours in 5 minute steps.
    The \setting{Sleep Timer} is reset on boot.
    \opt{alarm}{Using this option disables the \setting{Wake up alarm}.}
}

\item[Debug (Keep Out!):]
  This sub menu is intended to be used \emph{only} by Rockbox developers.
  It shows hardware, disk, battery status and other technical information.  
  \warn{It is not recommended that users access this menu unless instructed to
  do so in the course of fixing a problem with Rockbox. If you think you have 
  messed up your settings by use of this menu please try to reset \emph{all} 
  settings before asking for help.}
\end{description}

\opt{player}{
  \section{Shutdown}
  This menu option saves the Rockbox configuration and turns off the hard
  drive before shutting down the machine. For maximum safety this procedure
  is recommended when turning off the \dap. (There is a very small risk
  of hard disk corruption otherwise.) See \reference{ref:Safeshutdown}
  for more details.
}

\opt{quickscreen}
{
\section{\label{ref:QuickScreen}Quick Screen}
  Although the \setting{Quick Screen} is accessible from nearly everywhere,
  not just the \setting{Main Menu}, it is worth mentioning here.  It allows
  rapid access to your four favourite settings.  The default settings are
  \setting{Shuffle} (\reference{ref:PlaybackSettings}),
  \setting{Repeat} (\reference{ref:PlaybackSettings}) and the
  \setting{Show Files} (\reference{ref:ShowFiles}) options, but almost all
  configurable options in Rockbox can be placed on this screen.  To change the
  options, navigate through the menus to the setting you want to add and press
  \ActionStdContext.  In the menu which appears you will be given options
  to place the setting on the \setting{Quick Screen}.
  
  Press \ActionStdQuickScreen{} to access it and \ActionQuickScreenExit{} to exit.
  The direction buttons will modify the individual setting values as indicated 
  by the arrow icons. Please note that the settings at opposite sides of the
   screen cycle through the available options in opposite directions.
   Therefore if you select the same setting at e.g. the top and bottom of the
   quickscreen, then pressing up and down will cycle through this setting in
   opposite directions.
}
