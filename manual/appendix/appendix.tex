% $Id$ %
\appendix

% $Id$ %
\chapter{File formats}
\section{\label{ref:Supportedfileformats}Supported file formats}
\begin{table}
\begin{rbtabular}{\textwidth}{clp{7em}X}%
{\textbf{Icon} & \textbf{File Type} & \textbf{Extension} 
  & \textbf{Action when selected}}{}{}
\includegraphics[width=0.37cm]{appendix/images/icon-directory.png} 
  & Directory & \emph{none} & Enter the directory \\
\opt{recorder,recorderv2fm,ondiofm,ondiosp}{
  \includegraphics[width=0.37cm]{appendix/images/icon-rolo.png} 
  & Rockbox firmware & \fname{.ajz} & Load the new firmware with ROLO \\
}
\opt{swcodec}{
  \includegraphics[width=0.37cm]{appendix/images/icon-audio-file.png} 
  & Audio file & \emph{various}\newline%
  (see \ref{ref:Supportedaudioformats})%
  % do NOT use \reference{} here as that will break the table.
  & Start playing the file and show the WPS\\
}
  & Bookmark & \fname{.bmark} & Display all bookmarks for an audio file\\
\opt{lcd_bitmap}{
  & Game of Life & \fname{.cells} & Show the configuration with the
     ``Rocklife'' plugin\\
}
\includegraphics[width=0.37cm]{appendix/images/icon-config.png} 
  & Configuration File & \fname{.cfg} & Load the settings file\\
\includegraphics[width=0.37cm]{appendix/images/icon-chip8.png} 
  & Chip8 game & \fname{.ch8} & Play the Chip8 game \\
\opt{lcd_color}{
  & Colours & \fname{.colours} & Open the colours file for editing.
    See \reference{ref:ChangingFiletypeColours}.\\
}
\includegraphics[width=0.37cm]{appendix/images/icon-cuesheet.png} 
  & Cuesheet & \fname{.cue} & View the cuesheet file \\
\opt{radio}{
  & FM Presets & \fname{.fmr} & Load the FM Presets (previous are discarded)\\
}
\includegraphics[width=0.37cm]{appendix/images/icon-font.png} 
  & Font & \fname{.fnt} & Change the user interface font to this one\\
\opt{gigabeat}{
  \includegraphics[width=0.37cm]{appendix/images/icon-rolo.png} 
  & Rockbox firmware & \fname{.gigabeat} & Load the new firmware with ROLO \\
}
\opt{iaudio}{
  \includegraphics[width=0.37cm]{appendix/images/icon-rolo.png} 
  & Rockbox firmware & \fname{.iaudio} & Load the new firmware with ROLO \\
}
\opt{ipod}{
  \includegraphics[width=0.37cm]{appendix/images/icon-rolo.png} 
  & Rockbox firmware & \fname{.ipod} & Load the new firmware with ROLO \\
}
\opt{h100,h300}{
  \includegraphics[width=0.37cm]{appendix/images/icon-rolo.png} 
  & Rockbox firmware & \fname{.iriver} & Load the new firmware with ROLO \\
}
\includegraphics[width=0.37cm]{appendix/images/icon-image-file.png} 
  & Image & \fname{.jpg} & View the JPEG image \\
  & Link & \fname{.link} & Display list of target files and directories;
    selecting one jumps to the target. See \reference{ref:Shortcutsplugin}.\\
\includegraphics[width=0.37cm]{appendix/images/icon-lang.png} 
  & Language File & \fname{.lng} & Load the language file \\
\includegraphics[width=0.37cm]{appendix/images/icon-playlist.png}
  & Playlist & \fname{.m3u, .m3u8} & Load the playlist and start playing 
    the first file \\
\opt{h10,h10_5gb,sansa}{
  \includegraphics[width=0.37cm]{appendix/images/icon-rolo.png} 
  & Rockbox firmware & \fname{.mi4} & Load the new firmware with ROLO \\
}
\opt{player}{
  \includegraphics[width=0.37cm]{appendix/images/icon-rolo.png} 
  & Rockbox firmware & \fname{.mod} & Load the new firmware with ROLO \\
}
\opt{masd,masf}{
  \includegraphics[width=0.37cm]{appendix/images/icon-audio-file.png} 
  & Audio file & \fname{.mp2, .mp3} & Start playing the file and show the WPS\\
}
\opt{swcodec}{
 \includegraphics[width=0.37cm]{appendix/images/icon-movie-file.png}
 & Video & \fname{.mpg, .mpeg, .mpv, .m2v} & Play the MPEG1/2 video \\
}
\includegraphics[width=0.37cm]{appendix/images/icon-rock.png} 
  & Plugin & \fname{.rock} & Start the plugin\\
\opt{masf}{\opt{lcd_bitmap}{
  \includegraphics[width=0.37cm]{appendix/images/icon-movie-file.png} 
    & Rockbox Video & \fname{.rvf} & View the movie (Rockbox format)\\}
}
\opt{sansaAMS}{
  \includegraphics[width=0.37cm]{appendix/images/icon-rolo.png} 
  & Rockbox firmware & \fname{.sansa} & Load the new firmware with ROLO \\
}
\includegraphics[width=0.37cm]{appendix/images/icon-text.png} 
  & Text File & \fname{.txt} & Display the text file using the text viewer plugin\\
\opt{archos}{
  \includegraphics[width=0.37cm]{appendix/images/icon-ucl.png} 
    & Flash Image & \fname{.ucl} & Flash the Rockbox image into the ROM \\
  }
  & Voice file & \fname{.voice} & Allow Rockbox to speak menus\\
\opt{masf}{
  \includegraphics[width=0.37cm]{appendix/images/icon-wav-file.png} 
    & Wave Audio File & \fname{.wav} & Play the WAV file \\%
}
\includegraphics[width=0.37cm]{appendix/images/icon-wps.png} 
  & While Playing Screen & \fname{.wps} & Load the new WPS display configuration\\
\end{rbtabular}
\end{table}

\opt{swcodec}{
  \section{\label{ref:Supportedaudioformats}Supported audio formats}
  \subsection{Lossy Codecs}
  \begin{rbtabular}{\textwidth}{lp{6em}X}%
  {\textbf{Format} & \textbf{Extension} & \textbf{Notes}}{}{}
    ATSC A/52 & \fname{.a52, .ac3} & Supports downmixing for playback of 5.1 streams in stereo. \\
    ADX & \fname{.adx} & \\
    Advanced Audio Coding & \fname{.m4a, .m4b, .mp4} & \\
    MPEG audio & \fname{.mp1, .mpa, .mp2, .mp3} & \\
    Musepack & \fname{.mpc} & Supports SV7 and SV8 in mono/stereo \\
    OGG/Vorbis & \fname{.ogg, .oga} & Some old ``floor 0'' files may crash Rockbox. \\
    Sony Audio & \fname{.oma, .aa3} & \\
    RealAudio & \fname{.rm, .ra, .rmvb} & \\
    Speex & \fname{.spx} & \\
    Dialogic telephony type & \fname{.vox} & \\
    Windows Media Audio & \fname{.wma, .wmv, .asf} & \\
  \end{rbtabular}

  \subsection{Lossless Codecs}
  \begin{rbtabular}{\textwidth}{lp{6em}X}%
  {\textbf{Format} & \textbf{Extension} & \textbf{Notes}}{}{}
    Audio Interchange File Format & \fname{.aif, .aiff} & AIFF supports following formats: \\
                                  &                     & linear pcm 8/16/24/32 bit. \\
                                  &                     & IEEE float 32/64 bit\\
                                  &                     & ITU-T G.711 a-low/$\mu$-low\\
                                  &                     & QuickTime IMA ADPCM\\
    \opt{h100,h300,x5,m5,m3}{
      Monkey's Audio & \fname{.ape, .mac} & -c1000 and -c2000 files decode
      fast enough to be useful.}
    \opt{gigabeatf}{
      Monkey's Audio & \fname{.ape, .mac} & -c1000 to -c3000
      files decode fast enough to be useful.}
    \opt{gigabeats}{
      Monkey's Audio & \fname{.ape, .mac} & -c1000 to -c4000 files decode
      fast enough to be useful.}
    \opt{ipod,h10,h10_5gb,mrobe100,sansa}{
      Monkey's Audio & \fname{.ape, .mac} & Only -c1000 files decode fast
      enough to be useful.}
    \\
    Sun Audio & \fname{.au, .snd} & Sun Audio supports following formats: \\
              &                   & linear pcm 8/16/24/32 bit. \\
              &                   & IEEE float 32/64 bit\\
              &                   & ITU-T G.711 a-low/$\mu$-low\\
    Free Lossless Audio & \fname{.flac} & \\
    Apple Lossless & \fname{.m4a, .mp4} & \\
    Shorten & \fname{.shn} & Seeking not supported.\\
    Wave64 & \fname{.w64} & Wave/Wave64 supports following formats: \\
    Waveform audio format & \fname{.wav} & linear pcm 8/16/24/32 bit. \\
                          &              & IEEE float 32/64 bit\\
                          &              & ITU-T G.711 a-low/$\mu$-low\\
                          &              & Microsoft ADPCM\\
                          &              & Intel DVI ADPCM(IMA ADPCM) 2/3/4/5 bit\\
                          &              & Dialogic OKI ADPCM\\
                          &              & YAMAHA ADPCM\\
                          &              & Adobe SWF ADPCM\\
    Wavpack & \fname{.wv} & \\
  \end{rbtabular}

  \subsection{Other Codecs}
  \begin{rbtabular}{\textwidth}{lp{6em}X}%
  {\textbf{Format} & \textbf{Extension} & \textbf{Notes}}{}{}
    Atari Sound Format & \fname{.cmc, .cm3, .cmr, .cms, .dmc, .dlt, .mpt, .mpd} & \\
    Synthetic music Mobile Application Format & \fname{.mmf} & Supports PCM/ADPCM only \\
    MOD & \fname{.mod} & \\
    NES Sound Format & \fname{.nsf, .nsfe} & \\
    Atari SAP & \fname{.sap} & \\
    Sound Interface Device & \fname{.sid} & \\
    SPC700 & \fname{.spc} & \\
  \end{rbtabular}
}


% $Id$ %
\chapter{\label{ref:wps_tags}WPS Tags}
\section{Status Bar}
\begin{table}
\begin{tagmap}{}{}
\config{\%we} & Display Status Bar\\
\config{\%wd} & Hide Status Bar\\
\end{tagmap}
\end{table}
These tags override the player setting for the display of the status bar.
They must be noted on their own line (which will not be shown in the WPS).

\section{Information from the track tags}
\begin{table}
  \begin{tagmap}{}{}
    \config{\%ia} & Artist\\
    \config{\%ic} & Composer\\
    \config{\%iA} & Album Artist\\
    \config{\%id} & Album Name\\
    \config{\%iG} & Grouping\\
    \config{\%ig} & Genre Name\\
    \config{\%in} & Track Number\\
    \config{\%it} & Track Title\\
    \config{\%iC} & Comment\\
    \config{\%iv} & ID3 version (1.0, 1.1, 2.2, 2.3, 2.4, or empty if not an ID3 tag)\\
    \config{\%iy} & Year\\
    \config{\%ik} & Disc Number\\
  \end{tagmap}
\end{table}
Remember that this information is not always available, so use the 
conditionals to show alternate information in preference to assuming.

These tags, when written with a capital ``I'' (e.g. \config{\%Ia} or \config{\%Ic}),
show the information for the next song to be played.

\nopt{lcd_charcell}{
  \section{Viewports}
  \begin{table}
    \begin{tagmap}{}{}
      \nopt{lcd_non-mono}{~%
        \config{\%V{\textbar}x{\textbar}y{\textbar}[width]{\textbar}%
        [height]{\textbar}[font]{\textbar}}
        & (see section \ref{ref:Viewports})\\}

      \nopt{lcd_color}{\opt{lcd_non-mono}{%
        \config{\%V{\textbar}x{\textbar}y{\textbar}[width]{\textbar}%
        [height]{\textbar}[font]{\textbar}[fgshade]{\textbar}[bgshade]{\textbar}}
        & (see section \ref{ref:Viewports})\\}}

      \opt{lcd_color}{%
        \config{\%V{\textbar}x{\textbar}y{\textbar}[width]{\textbar}%
        [height]{\textbar}[font]{\textbar}[fgcolour]{\textbar}[bgcolour]{\textbar}}
        & (see section \ref{ref:Viewports})\\}

      \config{\%Vd'identifier'} & Display the 'identifier' viewport. E.g.
      \config{\%?C{\textless}\%C\%Vda{\textbar}\%Vdb{\textgreater}}
      will show viewport 'a' if album art is found, and 'b' if it isn't.\\
    \end{tagmap}
  \end{table}
}

\section{Power Related Information}
\begin{table}
  \begin{tagmap}{}{}
    \config{\%bl} & Numeric battery level in percents\\
                  & Can also be used in a conditional: 
                    \config{\%?bl{\textless}-1{\textbar}0{\textbar}1{\textbar}%
                    2{\textbar}\ldots{\textbar}N{\textgreater}},
                    where the value $-1$ is used when the battery level isn't
                    known (it usually is)\\
    \config{\%bv} & The battery level in volts\\
    \config{\%bt} & Estimated battery time left\\
    \config{\%bp} & ``p'' if the charger is connected (only on targets
                    that can charge batteries)\\
    \config{\%bc} & ``c'' if the unit is currently charging the battery (only on
                    targets that have software charge control or monitoring)\\
    \config{\%bs} & Remaining time of the sleep timer (if it is set)\\
  \end{tagmap}
\end{table}

\section{Information about the file}
\begin{table}
  \begin{tagmap}{}{}
    \config{\%fb} & File Bitrate (in kbps)\\
    \config{\%fc} & File Codec (e.g. ``MP3'' or ``FLAC''). %
           This tag can also be used in a conditional tag, %
           \config{\%?fc{\textless}mp1\-{\textbar}mp2\-{\textbar}mp3\-%
           {\textbar}aiff\-{\textbar}wav\-{\textbar}ogg\-{\textbar}flac\-%
           {\textbar}mpc\-{\textbar}a52\-{\textbar}wavpack\-{\textbar}alac\-%
           {\textbar}aac\-{\textbar}shn\-{\textbar}sid\-{\textbar}adx\-%
           {\textbar}nsf\-{\textbar}speex\-{\textbar}spc\-{\textbar}ape\-%
           {\textbar}wma\-{\textbar}mod\-{\textbar}sap%
           {\textbar}unknown{\textgreater}}.\\
                  & The codec order is as follows: MP1, MP2, MP3, AIFF, WAV,
           Ogg Vorbis (OGG), FLAC, MPC, AC3, WavPack (WV), ALAC, AAC,
           Shorten (SHN), SID, ADX, NSF, Speex, SPC, APE, WMA, MOD, SAP.\\
    \config{\%ff} & File Frequency (in Hz)\\
    \config{\%fk} & File Frequency (in KHz)\\
    \config{\%fm} & File Name\\
    \config{\%fn} & File Name (without extension)\\
    \config{\%fp} & File Path\\
    \config{\%fs} & File Size (in Kilobytes)\\
    \config{\%fv} & ``(avg)'' if variable bit rate or empty string if constant bit rate\\
    \config{\%d1} & First directory from the end of the file path\\
    \config{\%d2} & Second directory from the end of the file path\\
    \config{\%d3} & Third directory from the end of the file path\\
  \end{tagmap}
\end{table}
Example for the \config{\%dN} commands: If the path is 
``/Rock/Kent/Isola/11 - 747.mp3'', \config{\%d1} is ``Isola'', 
\config{\%d2} is ``Kent'' and \config{\%d3} is ``Rock''.

These tags, when written with the first letter capitalized (e.g. \config{\%Fn} or \config{\%D2}),
produce the information for the next file to be played.

\section{Playlist/Song Info}
\begin{table}
  \begin{tagmap}{}{}
    \config{\%pb} & Progress Bar\\
    \opt{player}{
          & This will display a one character ``cup''
            that empties as the time progresses.}
    \opt{lcd_bitmap}{
         & This will replace the entire line with a progress bar. \\
         & You can set the position, width and height of the progressbar %
           (in pixels) and load a custom image for it: %
           \config{\%pb{\textbar}image.bmp{\textbar}x{\textbar}y{\textbar}width{\textbar}height{\textbar}}} \\
    \opt{player}{%
    \config{\%pf} & Full-line progress bar \& time display\\
    }%
    \config{\%px} & Percentage Played In Song\\
    \config{\%pc} & Current Time In Song\\
    \config{\%pe} & Total Number of Playlist Entries\\
    \nopt{player}{%
    \config{\%pm} & Peak Meter. The entire line is used as %
                    volume peak meter.\\%
    }%
    \config{\%pn} & Playlist Name (without path or extension)\\
    \config{\%pp} & Playlist Position\\
    \config{\%pr} & Remaining Time In Song\\
    \config{\%ps} & ``s'' if shuffle mode is enabled\\
    \config{\%pt} & Total Track Time\\
    \config{\%pv} & Current volume (in dB). Can also be used in a conditional: \\
         & \config{\%?pv{\textless}0{\textbar}1{\textbar}2{\textbar}\ldots%
           {\textbar}N{\textgreater}}\\
         & 0 is used for mute, the last option is used for values greater than zero.\\
    \config{\%Sp} & Current Playback Pitch\\
  \end{tagmap}
\end{table}

\section{Runtime Database}
\begin{table}
  \begin{tagmap}{}{}
    \config{\%rp} & Song playcount\\
    \config{\%rr} & Song rating (0-10). This tag can also be used in a conditional tag: %
           \config{\%?rr{\textless}0{\textbar}1{\textbar}2{\textbar}3{\textbar}%
           4{\textbar}5{\textbar}6{\textbar}7{\textbar}8{\textbar}9{\textbar}%
           10{\textgreater}}\\
    \config{\%ra} & Autoscore for the song\\
  \end{tagmap}
\end{table}

\opt{swcodec}{
\section{Sound (DSP) settings}
\begin{table}
  \begin{tagmap}{}{}
    \config{\%Sp} & Current playback pitch \\
  \opt{swcodec}{
    \config{\%xf} & Crossfade setting, in the order: Off, Auto Skip, Man Skip,
           Shuffle, Shuffle and Man Skip, Always\\
    \config{\%rg} & ReplayGain value in use (x.y dB). If used as a conditional,
           Replaygain type in use: \config{\%?rg{\textless}Off{\textbar}Track%
           {\textbar}Album{\textbar}TrackShuffle{\textbar}AlbumShuffle%
           {\textbar}No tag{\textgreater}}\\
  }
  \end{tagmap}
\end{table}
}

% this will not include the "remote hold switch" tag for targets lacking 
% a main unit hold switch
\opt{hold_button}{
  \opt{remote_button_hold}{
    \section{Hold Switches}
    \begin{table}
      \begin{tagmap}{}{}
        \config{\%mh} & ``h'' if the main unit hold switch is on\\
        \config{\%mr} & ``r'' if the remote hold switch is on\\
      \end{tagmap}
    \end{table}
  }
  \nopt{remote_button_hold}{
    \section{Hold Switch}
    \begin{table}
      \begin{tagmap}{}{}
        \config{\%mh} & ``h'' if the hold switch is on\\
      \end{tagmap}
    \end{table}
  }
}

\section{Virtual LED}
\begin{table}
  \begin{tagmap}{}{}
    \config{\%lh} & ``h'' if the \disk\ is accessed\\
  \end{tagmap}
\end{table}

\section{Repeat Mode}
\begin{table}
  \begin{tagmap}{}{}
    \config{\%mm} & Repeat mode, 0-4, in the order: Off, All, One, Shuffle
           \opt{player,recorder,recorderv2fm}{, A-B}\\
  \end{tagmap}
\end{table}
Example: \config{\%?mm{\textless}Off{\textbar}All{\textbar}One{\textbar}Shuffle%
{\textbar}A-B{\textgreater}}

\section{Playback Mode}
\begin{table}
  \begin{tagmap}{}{}
    \config{\%mp} & Play status, 0-4, in the order: Stop, Play, Pause, 
           Fast Forward, Rewind\\
  \end{tagmap}
\end{table}
Example: \config{\%?mp{\textless}Stop{\textbar}Play{\textbar}Pause{\textbar}%
Ffwd{\textbar}Rew{\textgreater}}

\section{Current Screen}
\begin{table}
  \begin{tagmap}{}{}
    \config{\%cs} & The current screen, 1-5, in the order:
                Menus, WPS, Recording screen, FM Radio screen, Current Playlist screen\\
  \end{tagmap}
\end{table}
The tag can also be used as the switch in a conditional tag. For players without
some capabilities (e.g. having no FM radio) some values will be never yielded.

Example: \config{You are in the \%?cs{\textless}Main menu{\textbar}WPS{\textbar}%
Recording screen{\textbar}FM Radio screen{\textgreater}}

\section{Changing Volume}
\begin{table}
  \begin{tagmap}{}{}
    \config{\%mv[t]} & ``v'' if the volume is being changed\\
  \end{tagmap}
\end{table}

The tag produces the letter ``v'' while the volume is being changed and some
amount of time after that, i.e. after the volume button has been released. The
optional parameter \config{t} specifies that amount of time, in seconds. If it
is not specified, 1 second is assumed.

The tag can be used as the switch in a conditional tag to display different things
depending on whether the volume is being changed. It can produce neat effects
when used with conditional viewports.

Example: \config{\%?mv2.5{\textless}Volume changing{\textbar}\%pv{\textgreater}}

The example above will display the text ``Volume changing'' if the volume is
being changed and 2.5 secs after the volume button has been released. After
that, it will display the volume value.

\section{Settings}
\begin{table}
  \begin{tagmap}{}{}
    \config{\%St{\textbar}<setting name>{\textbar}} & The value of the Rockbox
             setting with the specified name. See \reference{ref:config_file_options}
             for the list of the available settings.\\
  \end{tagmap}
\end{table}

Examples:
\begin{enumerate}
\item As a simple tag: \config{\%St{\textbar}skip length{\textbar}}
\item As a conditional: \config{\%?St{\textbar}eq enabled{\textbar}{\textless}Eq is enabled{\textbar}Eq is disabled{\textgreater}}
\end{enumerate}


\opt{lcd_bitmap}{
\section{\label{ref:wps_images}Images}
\begin{table}
  \begin{tagmap}{}{}
    \nopt{archos}{%
    \config{\%X{\textbar}filename.bmp{\textbar}}
        & Load and set a backdrop image for the WPS.
          This image must be exactly the same size as your LCD.\\
    }%
    \config{\%x{\textbar}n{\textbar}filename{\textbar}x{\textbar}y{\textbar}}
        & Load and display an image\\
        & \config{n}: image ID (a-z and A-Z) for later referencing in \config{\%xd}\\
        & \config{filename}: file name relative to \fname{/.rockbox/} and including ``.bmp''\\
        & \config{x}: x coordinate\\
        & \config{y}: y coordinate.\\
    \config{\%xl{\textbar}n{\textbar}filename{\textbar}x{\textbar}y{\textbar}[nimages{\textbar}]}
        & Preload an image for later display (useful for when your images are displayed conditionally)\\
        & \config{n}: image ID (a-z and A-Z) for later referencing in \config{\%xd}\\
        & \config{filename}: file name relative to \fname{/.rockbox/} and including ``.bmp''\\
        & \config{x}: x coordinate\\
        & \config{y}: y coordinate\\
        & \config{nimages}: (optional) number of sub-images (tiled vertically, of the same height)
          contained in the bitmap. Default is 1.\\
    \config{\%xdn[i]} & Display a preloaded image\\
        & \config{n}: image ID (a-z and A-Z) as it was specified in \config{\%x} or \config{\%xl}\\
        & \config{i}: (optional) number of the sub-image to display (a-z for 1-26 and A-Z for 27-52).
          By default the first (i.e. top most) sub-image will be used.\\
  \end{tagmap}
\end{table}

Examples:
\begin{enumerate}
\item Load and display the image \fname{/.rockbox/bg.bmp} with ID ``a'' at 37, 109:\\
\config{\%x{\textbar}a{\textbar}bg.bmp{\textbar}37{\textbar}109{\textbar}}
\item Load a bitmap strip containing 5 volume icon images (all the same size)
with image ID ``M'', and then reference the individual sub-images in a conditional:\\
\config{\%xl{\textbar}M{\textbar}volume.bmp{\textbar}134{\textbar}153{\textbar}5{\textbar}}\\
\config{\%?pv<\%xdMa{\textbar}\%xdMb{\textbar}\%xdMc{\textbar}\%xdMd{\textbar}%
\%xdMe>}
\end{enumerate}


\note{
  \begin{itemize}
  \item The images must be in BMP format
  \item The image tag must be on its own line
  \item The ID is case sensitive, giving 52 different ID's
  \item The size of the LCD screen for each player varies. See table below 
        for appropriate sizes of each device. The x and y coordinates must 
        repect each of the players' limits.
  \end{itemize}
}
}

\opt{albumart}{
\section{Album Art}
\begin{table}
  \begin{tagmap}{}{}
    \config{\%Cl{\textbar}x{\textbar}y{\textbar}[[l{\textbar}c{\textbar}r]maxwidth]{\textbar}[[t{\textbar}c{\textbar}b]maxheight]{\textbar}}
        & Define the settings for albumart\\
        & \config{x}: x coordinate\\
        & \config{y}: y coordinate\\
        & \config{maxwidth}: Maximum height\\
        & \config{maxheight}: Maximum width\\
    \config{\%C}  & Display the album art as configured. This tag can also be used as a conditional.\\
  \end{tagmap}
\end{table}

The picture will be rescaled, preserving aspect ratio to fit the given
\config{maxwidth} and \config{maxheight}. If the aspect ratio doesn't match the
configured values, the picture will be placed according to the flags to the
\config{maxwidth} and \config{maxheight} parameters:
\begin{itemize}
  \item \config{maxwidth}:
    \begin{description}
      \item[\config{l}.] Align left
      \item[\config{c}.] Align centre (default)
      \item[\config{r}.] Align right
    \end{description}
  \item \config{maxheight}:
    \begin{description}
      \item[\config{t}.] Align top
      \item[\config{c}.] Align centre (default)
      \item[\config{b}.] Align bottom
    \end{description}
\end{itemize}

Examples:
\begin{enumerate}
  \item Load albumart at position 20,40 and display it without resizing:\\
  \config{\%Cl{\textbar}20{\textbar}40{\textbar}{\textbar}{\textbar}}
  \item Load albumart at position 0,20 and resize it to be at most 100x100
        pixels. If the image isn't square, align it to the bottom-right
        corner:\\
  \config{\%CL{\textbar}0{\textbar}20{\textbar}r100{\textbar}b100{\textbar}}
\end{enumerate}
}

\section{Alignment}
\begin{table}
  \begin{tagmap}{}{}
    \config{\%al} & Align the text left\\
    \config{\%aL} & Align the text left, or to the right if RTL language is in use\\
    \config{\%ac} & Centre the text\\
    \config{\%ar} & Align the text right\\
    \config{\%aR} & Align the text right, or to the left if RTL language is in use\\
  \end{tagmap}
\end{table}
All alignment tags may be present in one line, but they need to be in the 
order left -- centre -- right. If the aligned texts overlap, they are merged.

\section{Conditional Tags}

\begin{table}
\begin{tagmap}{}{}
\config{\%?xx{\textless}true{\textbar}false{\textgreater}}
    & If / Else: Evaluate for true or false case \\
\config{\%?xx{\textless}alt1{\textbar}alt2{\textbar}alt3{\textbar}\dots{\textbar}else{\textgreater}}
    & Enumerations: Evaluate for first / second / third / \dots / last condition \\
\end{tagmap}
\end{table}

\section{Subline Tags}

\begin{table}
\begin{tagmap}{}{}
\config{\%t{\textless}time{\textgreater}}
    & Set the subline display cycle time (\%t5 or \%t3.4 formats) \\
\config{;}
    & Split items on a line into separate sublines \\
\end{tagmap}
\end{table}

Allows grouping of several items (sublines) onto one line, with the
display cycling round the defined sublines. See
\reference{ref:AlternatingSublines} for details. 


\section{Time and Date}
  \begin{table}
    \begin{tagmap}{}{}
    \opt{rtc}{
      \config{\%cd}          & Day of month from 01 to 31\\
      \config{\%ce}          & Zero padded day of month from 1 to 31\\
      \config{\%cf}          & A conditional for 12/24 hour format. \%?cf{\textless}24 hour stuff{\textbar}12 hour stuff{\textgreater}\\
      \config{\%cH}          & Zero padded hour from 00 to 23 (24 hour format)\\
      \config{\%ck}          & Hour from 0 to 23 (24 hour format)\\
      \config{\%cI}          & Zero padded hour from 01 to 12 (am/pm format)\\
      \config{\%cl}          & Hour from 1 to 12 (am/pm format)\\
      \config{\%cm}          & Month from 01 to 12\\
      \config{\%cM}          & Minutes\\
      \config{\%cS}          & Seconds\\
      \config{\%cy}          & 2-digit year\\
      \config{\%cY}          & 4-digit year\\
      \config{\%cP}          & Capital AM/PM\\
      \config{\%cp}          & Lowercase am/pm\\
      \config{\%ca}          & Weekday name\\
      \config{\%cb}          & Month name\\
      \config{\%cu}          & Day of week from 1 to 7, 1 is Monday\\
      \config{\%cw}          & Day of week from 0 to 6, 0 is Sunday\\
    }
      \config{\%cc}          & Check for presence of the clock hardware\\
    \end{tagmap}
  \end{table}
The \%cc tag returns ``c'' if the necessary hardware is present and can also be
used as a conditional. This can be very useful for designing a WPS that works on
multiple targets, some with and some without a clock. By using this tag as a
conditional it is possible to display current date and time on those targets that
support this
\opt{rtc}{ (like the \playertype)},
or alternate information on those that do not
\nopt{rtc}{ (like the \playertype)}%
.

Example:
\config{\%?cc{\textless}\%cH:\%cM{\textbar}No clock detected{\textgreater}}


\section{Other Tags}
\begin{table}
\begin{tagmap}{}{}
  \config{\%\%}          & The character `\%'\\
  \config{\%{\textless}} & The character `{\textless}'\\
  \config{\%{\textbar}}  & The character `{\textbar}'\\
  \config{\%{\textgreater}} & The character `{\textgreater}'\\
  \config{\%;}           & The character `;'\\
  \config{\%s}           & Indicate that the line should scroll. Can occur 
                           anywhere in a line (given that the text is 
                           displayed; see conditionals above). You can specify 
                           up to ten scrolling lines. Scrolling lines can not 
                           contain dynamic content such as timers, peak meters 
                           or progress bars.\\
\end{tagmap}
\end{table}


\opt{lcd_bitmap}{\opt{tagcache}{
\input{appendix/album_art.tex}
}}

% $Id$ %
\chapter{\label{ref:config_file_options}Config file options}
\begin{center}
% define a local version of endhead, as using the output distinction adds
% an unwanted newline. endhead breaks with htlatex so we need to remove it
% for the html output.
\ifpdfoutput{\newcommand{\localendhead}{\endhead}}%
    {\newcommand{\localendhead}{}}
  \rowcolors{1}{tblevenrowbgcolor}{tbloddrowbgcolor}
  \begin{longtable}{@{}>{\raggedright}p{.35\textwidth}@{}>{\raggedright}p{.4\textwidth}@{}p{.25\textwidth}@{}}
    \toprule
    \rowcolor{white} \textbf{Setting} & \textbf{Allowed Values} & \textbf{Unit}\\
    \midrule\localendhead % endhead breaks with htlatex
    volume      & \opt{masd}{-78 to +18}%
                  \opt{masf}{-100 -to +12}%
                  \opt{h100,h300}{-84 to 0}%
                  \opt{ipodnano}{-72 to +6}%
                  \opt{ipodnano2g}{-74 to +6}%
                  \opt{ipodvideo}{-57 to +6}%
                  \opt{x5}{-73 to +6}
                  \opt{e200,e200v2}{-74 to +6}
                  \opt{ipodcolor}{-74 to +6}%
                                        & dB\\
    \nopt{x5}{%
      bass      & \opt{masd}{-15 to +15}%
                  \opt{masf}{-12 to +12}%
                  \opt{h100,h300}{0 to +24}%
                  \opt{ipod}{-6 to +9}%
                  \opt{e200,e200v2}{-24 to +24}%
                                        & dB\\
      treble    & \opt{masd}{-15 to +15}%
                  \opt{masf}{-12 to +12}%
                  \opt{h100,h300}{0 to +6}%
                  \opt{ipod}{-6 to +9}%
                  \opt{e200,e200v2}{-24 to +24}%
                                        & dB\\
    }%
    balance         & -100 to +100      & \%\\
    channels        & stereo, mono, custom, mono left, mono right, karaoke
                                        & N/A\\
    stereo\_width   & 0 to 250          & \%\\
    shuffle         & on, off               & N/A\\
    repeat          & off, all, one, shuffle, ab
                                        & N/A\\
    play selected   & on, off           & N/A\\
    party mode      & on, off           & N/A\\
    scan min step   & 1, 2, 3, 4, 5, 6, 8, 10, 15, 20, 25, 30, 45, 60
                                        & seconds\\
    seek acceleration & very fast, fast, normal, slow, very slow & N/A\\
    antiskip        & 5s, 15s, 30s, 1min, 2min, 3min, 5min, 10min & N/A\\
    volume fade     & on, off           & N/A\\
    sort case       & on, off           & N/A\\
    show files      & all, supported, music, playlists & N/A\\
    show filename exts & off, on, unknown, view\_all & N/A\\
    follow playlist & on, off           & N/A\\
    playlist viewer icons
                    & on, off           & N/A\\
    playlist viewer indices
                    & on, off           & N/A\\
    playlist viewer track display
                    & track name,full path
                                        & N/A\\
    recursive directory insert
                    & on, off, ask      & N/A\\
    scroll speed    & 1 to 25           & Hz\\
    scroll delay    & 0 to 2500         & ms\\
    scroll step     & \fixme{devise a way to get ranges from config-*.h} & pixels\\
    screen scroll step & \fixme{devise a way to get ranges from config-*.h} & pixels\\
    Screen Scrolls Out Of View & on, off & N/A\\
    bidir limit     & 0 to 200          & \% screen\\
    scroll paginated & on, off & N/A\\
    hold\_lr\_for\_scroll\_in\_list & on, off & N/A\\
    \opt{lcd_bitmap}{
      show path in browser & off, current directory, full path & N/A\\
    }
    contrast        & 0 to 63           & N/A\\
    backlight timeout
                    & off, on, 1, 2, 3, 4, 5, 6, 7, 8, 9, 10, 15, 20, 25, 30,
                      45, 60, 90, 120        & seconds\\
    backlight timeout plugged
                    & off, on, 1, 2, 3, 4, 5, 6, 7, 8, 9, 10, 15, 20, 25, 30,
                      45, 60, 90, 120        & seconds\\
    backlight filters first keypress & on, off & N/A\\
    backlight on button hold & normal, off, on & N/A\\
    caption backlight & on, off & N/A\\
    brightness      & \fixme{devise a way to get ranges from config-*.h} & N/A\\
    disk spindown   & 3 to 254          & seconds\\
    battery capacity & \fixme{devise a way to get ranges from config-*.h} & mAh\\
    \opt{battery_types}{
      battery type  & alkaline, nimh    & N/A\\
    }
    \opt{HAVE_CAR_ADAPTER_MODE}{
      car adapter mode & on, off & N/A\\
    }
    \opt{accessory_supply}{
      accessory power supply & on, off & N/A\\
    }
    \opt{usb_hid}{
        usb hid & on, off & N/A\\
        usb keypad mode
                    & multimedia, presentation, browser\opt{usb_hid_mouse}{, mouse}& N/A\\
    }
    idle poweroff   & off, 1, 2, 3, 4, 5, 6, 7, 8, 9, 10, 15, 30, 45, 60
                                        & minutes\\
    max files in playlist & 1000 - 32000 & N/A\\
    max files in dir & 50 - 10000       & N/A\\
    lang            & /path/filename.lng & N/A\\
    wps             & /path/filename.wps & N/A\\
    autocreate bookmarks
                    & off, on           & N/A\\
    autoload bookmarks
                    & off, on           & N/A\\
    use most-recent-bookmarks
                    & off, on           & N/A\\
    pause on headphone unplug & off, pause, pause and resume & N/A\\
    rewind duration on pause & 0 to 15  & seconds\\
    disable autoresume if phones not present & off, on & N/A\\
    Last.fm Logging & off, on           & N/A\\
    talk dir        & off, number, spell& N/A\\
    talk dir clip   & off, on           & N/A\\
    talk file       & off, number, spell& N/A\\
    talk file clip  & off, on           & N/A\\
    talk filetype   & off, on           & N/A\\
    talk menu       & off, on           & N/A\\
    Announce Battery Level & off, on    & N/A\\
    sort files      & alpha, oldest, newest, type & N/A\\
    sort dirs       & alpha, oldest, newest & N/A\\
    sort interpret number & digits, numbers & N/A\\
    tagcache\_autoupdate
                    & on, off           & N/A\\
    warn when erasing dynamic playlist
                    & on, off           & N/A\\
    cuesheet support
                    & on, off           & N/A\\
    folder navigation & off, on, random & N/A\\
    gather runtime data & off, on       & N/A\\
    \opt{usb_charging_enable}{
      usb charging  & on, off           & N/A\\
    }
    skip length     & outro, track, 1s, 2s, 3s, 5s, 7s, 10s, 15s, 20s, 1min,
            90s, 2min, 3min, 5min, 10min, 15min & N/A\\
    prevent track skip
                    & on, off           & N/A\\
    start in screen & previous, root, files, db, wps, menu,
      \opt{recording}{recording, }
      \opt{radio}{radio, }
      bookmarks                         & N/A\\
    playlist catalog directory & /path/to/dir & N/A\\
    \opt{wheel_acceleration}{
      list\_accel\_start\_delay & 0 to 10  & ms\\
      list\_accel\_wait        & 1 to 10  & seconds\\
    }
%
    \opt{swcodec}{
      replaygain type
                    & track, album, track shuffle, off
                                        & N/A\\
      replaygain noclip
                    & on, off           & N/A\\
      replaygain preamp
                    & -120 to 120       & 0.1dB\\
%
      \opt{crossfade}{
      crossfade     & off, auto track change, man track skip, shuffle,
                    shuffle or man track skip, always
                                        & N/A\\
      crossfade fade in delay
                    & 0 to 7            & seconds\\
      crossfade fade out delay
                    & 0 to 7            & seconds\\
      crossfade fade in duration
                    & 0 to 15           & seconds\\
      crossfade fade out duration
                    & 0 to 15           & seconds\\
      crossfade fade out mode
                    & crossfade, mix    & N/A\\
      }
%
      crossfeed     & on, off           & N/A\\
      crossfeed direct gain
                    & 0 to 60           & 0.1dB\\
      crossfeed cross gain
                    & 30 to 120         & 0.1dB\\
      crossfeed hf attenuation
                    & 60 to 240         & 0.1dB\\
      crossfeed hf cutoff
                    & 500 to 2000       & Hz\\
%
      eq enabled    & on, off           & N/A\\
      eq precut     & 0 to 240          & 0.1dB\\
      eq band 0 cutoff & 0 to 32768     & Hz\\
      eq band 1 cutoff & 0 to 32768     & Hz\\
      eq band 2 cutoff & 0 to 32768     & Hz\\
      eq band 3 cutoff & 0 to 32768     & Hz\\
      eq band 4 cutoff & 0 to 32768     & Hz\\
      eq band 0 q   & 0 to 64           & N/A\\
      eq band 1 q   & 0 to 64           & N/A\\
      eq band 2 q   & 0 to 64           & N/A\\
      eq band 3 q   & 0 to 64           & N/A\\
      eq band 4 q   & 0 to 64           & N/A\\
      eq band 0 gain & -240 to 240      & 0.1dB\\
      eq band 1 gain & -240 to 240      & 0.1dB\\
      eq band 2 gain & -240 to 240      & 0.1dB\\
      eq band 3 gain & -240 to 240      & 0.1dB\\
      eq band 4 gain & -240 to 240      & 0.1dB\\
%
      dithering enabled & on, off       & N/A\\
%
      timestretch enabled & on, off     & N/A\\
%
      compressor threshold      & 0 to -24      & -3dB\\
      compressor makeup gain    & off, auto     & N/A\\
      compressor ratio          & 2:1, 4:1, 6:1, 10:1, limit
                                                & N/A\\
      compressor knee           & hard knee, soft knee
                                                & N/A\\
      compressor release time   & 100 to 1000   & 100 ms\\
%
      beep          & off, weak, moderate, strong & N/A\\
      keyclick      & off, weak, moderate, strong & N/A\\
      keyclick repeats & on, off        & N/A\\
      dircache      & on, off           & N/A\\
      tagcache\_ram & on, off           & N/A\\
    }%

    \opt{masf}{
      loudness      & 0 to 17           & N/A\\
      superbass     & on, off           & N/A\\
      auto volume   & off, 20ms, 2s, 4s, 8s
                                        & seconds\\
      mdb enable    & on,off            & N/A\\
      mdb strength  & 0 to 127          & dB\\
      mdb harmonics & 0 to 100          & \%\\
      mdb center    & 20 to 300         & Hz\\
      mdb shape     & 50 to 300         & Hz\\
    }%

    \opt{lcd_bitmap}{
      peak meter release
                    & 1 to 126          & ?\\
      peak meter hold
                    & off, 200ms, 300ms, 500ms, 1, 2, 3, 4, 5, 6, 7, 8, 9, 10,
                      15, 20, 30, 1min  & N/A \\
      peak meter clip hold
                    & on, 1, 2, 3, 4, 5, 6, 7, 8, 9, 10, 15, 20, 25, 30, 45,
                      60, 90, 2min, 3min, 5min, 10min, 20min, 45min, 90min
                                        & N/A \\
      peak meter busy & on, off         & N/A\\
      peak meter dbfs & on, off         & on:~dbfs, off:~linear\\
      peak meter min  & 0 to 89 (dB) or 0 to 100 (\%)
                                        & dB or \%\\
      peak meter max  & 0 to 89 /(dB) or 0 to 100 (\%)
                                        & dB or \%\\
      statusbar     & off, top, bottom  & N/A\\
      \opt{remote}{
        remote statusbar & off, top, bottom & N/A\\
      }
      scrollbar     & off, left, right  & N/A\\
      scrollbar width & 3 to LCD width / 10 (\fixme{devise a way
                    to get ranges from config-*.h})& pixels\\
      volume display
                    & graphic, numeric  & N/A\\
      battery display
                    & graphic, numeric  & N/A\\
      font          & /path/filename.fnt & N/A\\
      kbd           & /path/filename.kbd & N/A\\
      \opt{lcd_invert}{
        invert        & on, off           & N/A\\
      }
      \opt{lcd_flip}{
        flip display  & on, off           & N/A\\
      }
      selector type   & pointer, bar (inverse)
        \opt{lcd_color}{, bar (color), bar (gradient)} & N/A\\
      show icons    & on, off           & N/A\\
      iconset       & /path/filename.bmp & N/A\\
      viewers iconset & /path/filename.bmp & N/A\\
    }%

    \opt{swcodec}{% This doesn't depend on swcodec but using a \nopt here
                  % causes ondiosp not to build for mysterious reasons.
      backdrop      & /path/filename.bmp    & N/A\\
    }%

    \opt{lcd_color}{
      foreground colour & 000000 to FFFFFF   & RRGGBB\\
      background colour & 000000 to FFFFFF   & RRGGBB\\
      line selector start colour & 000000 to FFFFFF  & RRGGBB\\
      line selector end colour   & 000000 to FFFFFF  & RRGGBB\\
      line selector text colour  & 000000 to FFFFFF  & RRGGBB\\
      filetype colours & /path/filename.colours & N/A\\
    }

    \opt{HAVE_REMOTE_LCD}{
      rwps      & /path/filename.rwps   & N/A\\
      remote contrast
                & 5 to 63               & N/A\\
      remote invert
                & on, off               & N/A\\
      remote flip display
                & on, off               & N/A\\
      remote backlight timeout
                & off, on, 1, 2, 3, 4, 5, 6, 7, 8, 9, 10, 15, 20, 25,
                  30, 45, 60, 90        & seconds\\
      remote backlight timeout plugged
                & off, on, 1, 2, 3, 4, 5, 6, 7, 8, 9, 10, 15, 20, 25,
                  30, 45, 60, 90        & seconds\\
      remote caption backlight
                & on, off               & N/A\\
      remote scroll speed
                & 0 to 15               & N/A\\
      remote scroll step
                & 1 to 160              & N/A\\
      remote scroll delay
                & 0 to 2500             & ms\\ 
      remote bidir limit
                & 0 to 200              & N/A\\
      backlight filters first remote keypress
                & on, off               & N/A\\
      remote iconset & /path/filename.bmp & N/A\\
      remote viewers iconset & /path/filename.bmp & N/A\\
      \opt{h100,h300}{
        remote reduce ticking
                & on, off               & N/A\\
      }%
    }
    \opt{rtc}{
      time format & 12hour, 24hour      & N/A\\
    }%
    \opt{recording}{
     rec quality & 0 to 7               & 0: small size, 7: high quality\\
     rec frequency
                & 48, 44, 32, 24, 22, 16 & kHz\\
     rec source & mic, line, spdif      & N/A\\
     rec channels & mono, stereo        & N/A\\
     rec mic gain & 0 to 15             & N/A\\
     rec left gain & 0 to 15            & N/A\\
     rec right gain
                & 0 to 15               & N/A\\
     editable recordings
                & off,on                & N/A\\
     rec timesplit
                & off, 0:05, 0:10, 0:15, 0:30, 1:00, 2:00, 4:00, 6:00,
                  8:00, 16:00, 24:00    & h:mm\\
     pre-recording time
                & off, 1 to 30          & seconds\\
     rec directory & /path/to/dir       & N/A\\
    }%
    \opt{spdif_power}{
      spdif enable & off, on            & N/A\\
    }%
    \opt{radio}{
      force fm mono
                & off, on               & N/A\\
    }%
    \opt{player}{
      jump scroll
                & 0 to 5                & N/A\\
      jump scroll delay
                & 0 to 250              & 0.01s\\
    }%

    \bottomrule
  \end{longtable}
\end{center}


\input{appendix/menu_structure.tex}

\chapter{User feedback}\label{sec:feedback}
\section{Bug reports}
If you experience inappropriate performance from any supported feature,
please file a bug report on our web page. Do not report missing
features as bugs, instead file them as feature ideas (see below).

For open bug reports refer to
\url{http://www.rockbox.org/tracker/index.php?type=2}

\subsection{Rules for submitting new bug reports}

\begin{enumerate}
\item  Check that the bug has not already been reported
\item  Always include the following information in your bug report:
\end{enumerate}

\begin{itemize}
\item  Which exact \dap{} you have.
\item  Which exact Rockbox version you are using
(Menu{}-{\textgreater}Info {}-{\textgreater} Version)
\item  A step{}-by{}-step description of what you did and what happened
\item  Whether the problem is repeatable or a one{}-time occurrence
\item  All relevant data regarding the problem, such as playlists, MP3
files etc. (IMPORTANT!) 
\end{itemize}

\section{Feature ideas}
To suggest an idea for a feature or to read those made by others, see
\url{http://forums.rockbox.org/index.php?board=49.0}.  Please keep in
mind that this forum is for the discussion of feature ideas - they are not
 requests and there is no guarantee they will be acted upon.

\subsection{Rules for submitting a new feature idea}

\begin{enumerate}
\item Check that the feature has not already been suggested. 
  Duplicates are really boring!
\item Check that the feature has not already been implemented. 
  Download the latest current/daily build and/or search the mail list archive.
\item Check that the feature is possible to implement (see \reference{ref:NODO}).
\end{enumerate}

\subsection{\label{ref:NODO}Features we will not implement}
This is a list of Feature Requests we get repeatedly that we simply
cannot do. View it as the opposite of a TODO!

\begin{itemize}
\opt{archos}{
\item Record to WAV (uncompressed) or MP3pro format.\\
The recording hardware (the MAS) does not allow us to do this
\item Crossfade between tracks.\\
  Crossfading would require two mp3 decoders, and we only have one. 
  This is not possible.
\item Support MP3pro, WMA or other sound format playback.\\
  The mp3{}-decoding hardware can only play MP3. We cannot make it play other 
  sound formats.
\item  Converting OGG $\rightarrow$ MP3.\\
  The mp3{}-decoding hardware cannot decode OGG. It can be reprogrammed, but 
  there is too little memory for OGG and we have no documentation on how to 
  program the MAS' DSP. Doing the conversion with the CPU is impossible, since 
  a 12MHz SH1 is far too slow for this daunting task.
\item Archos Multimedia support.\\
  The Archos Multimedia is a completely different beast. It is an entirely 
  different architecture, different CPU and upgrading the software is done 
  a completely different way. We do not wish to venture into this.  Others 
  may do so. We will not.
\item Multi{}-band (or graphic) equaliser.\\
  We cannot access information for that kind of visualisation from the MP3 
  decoding hardware.
\item CBR recording.\\
  The MP3 encoding hardware does not allow this.
\item Change tempo of a song without changing pitch.\\
  The MP3 decoding hardware does not allow this.
\item Graphic frequency (spectrum analyser).\\
  We cannot access the audio waveform from the MP3 decoder so we cannot analyse 
  it. Even if we had access to it, the CPU would probably be too slow to 
  perform the analysis anyway.
\item Cool sound effects.\\
  Adding new sound effects requires reprogramming the MAS chip, and we cannot 
  do that. The MAS chip is programmable, but we have no access to the chip 
  documentation.
}
\nopt{h300,x5}{
\item Interfacing with other USB devices (like cameras) or 2 player games over USB.\\
  The USB system demands that there is a master that talks to a slave. The
  \dap{} can only serve as a slave, as most other USB devices such as
  cameras can. Thus, without a master no communication between the slaves
  can take place. If that is not enough, we have no way of actually
  controlling the communication performed over USB since the USB circuit
  in the \dap{} is strictly made for disk{}-access and does not allow us
  to play with it the way we'd need for any good communication to work.
}
\item Support other file systems than FAT32 (like NTFS or ext2 etc.).\\
  No.
  \opt{archos}{Rockbox needs to support FAT32 since it can only start off a FAT32
  partition (since that is the only way the ROM can load it), and adding}%
  support for more file systems will just take away valuable ram for
  unnecessary features. You can partition your \dap{} fine, just make sure
  the first one is FAT32 and then make the other ones whatever file system
  you want. Just do not expect Rockbox to understand them.
\item Add scandisk{}-like features.\\
  It would be a very slow operation that would drain the batteries and
  take a lot of useful ram for something that is much better and faster
  done when connected to a host computer.
\item Alphabetical list skipping.\\
  Skipping around the lists by jumping letters (i.e skip all C's and go
  straight to the first D). This isn't feasible with the current list
  implementation, if you really want this you can get similar effects using
  the database (see \reference{ref:database}).
\item Add support for non standard tag formats.\\
APE tags in MP3 files has been rejected a few times already. Its not something we want.
\item Implementing the ability to playback DRM files.\\
  Firstly, this would be extremely difficult to implement legally - Rockbox
  is not legal entity as such, and therefore is unable to enter into license
  agreements with providers of DRM technology.
  Secondly, Rockbox is open source, which would mean that any DRM technology we
  incorporated into our codebase would suddenly become visible to the whole world,
  completely defeating its purpose. Remember, DRM achieves part of it's security
  through obscurity, and publishing the keys necessary to decrypt DRM'd 
  media would essentially render it useless.
\end{itemize}

\chapter{Credits}
People that have contributed to the project, one way or another. Friends!
%
\begin{multicols}{2}
\noindent\caps{\small{\input{CREDITS.tex}}}
\end{multicols}

\chapter{Licenses}

\section{GNU Free Documentation License}
\input{appendix/fdl.tex}
\newpage
\section{The GNU General Public License}
\input{appendix/gpl-2.0.tex}
